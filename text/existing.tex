\chapter{Existing frameworks and tools}

This chapter starts with an overview of existing frameworks for
implementing Zero Knowledge Proofs of Knowledge. Currently available
are:
\begin{itemize}
\item CACE Project compiler
\item ZKPDL, a library
\item Idemix
\end{itemize}
A detailed comparison of the frameworks follow.
The chapter then continues on describing ANTLR, a parser generator
tool producing LL(*) parsers.  Next on is LLVM, a compiler
infrastructure framework that has seen wide use recently in many
fields relating to compilers and computers in general. ANTLR and LLVM
will be used to make a custom compiler later on so some knowledge
about them is necessary. Next is GEZEL, a cycle-true cosimulation
environment which can also generate VHDL or Verilog.

\section{CACE Project}

The CACE Project compiler has the following typical flow (as depicted
in Figure \ref{fig:cace_workflow}):
\begin{enumerate}
\item Write PSL (Protocol Specification Language)
\item Generate PIL (Protocol Interface Language) from PSL
\item Generate C or Java code from PIL
\item (Optional) Generate LaTeX from PIL
\end{enumerate}

The PIL itself gives all the details of a protocol and is easily
understandable and exportable to LaTeX. The ease of understanding
the steps will allow better cryptographic analysis. The tool must
now provide an accurate and representable translation which does not
lose any properties.

\begin{figure}[hbt!]
\centering
  \begin{tikzpicture}[>=stealth,level distance=1.5cm,font=\tiny]
    \tikzstyle{edge from parent}=[draw,->]

    \Tree [.\node[language](psl){Protocol \\ Specification \\ Language (PSL)};
      [.\node[compiler](pc){Protocol \\ Compiler};
        [.\node[language](pil){Protocol \\ Implementation \\ Language (PIL)};
          [.\node[compiler](c){C}; \node[language](code){Code};]
          [.\node[compiler](latex){\LaTeX}; \node[language](doc){Documentation};]
        ]
      ]
    ]

    \node[compiler] (pvt)         [right=of pc,anchor=west]          {Protocol \\ Verification \\ Toolbox}
    child {node[language] {Proof of \\ Soundness}};

    \node[compiler] (sigma) [left=of pil.north west,anchor=center] {$\Sigma 2 N I Z K$};
    \node[compiler] (cost) [left=of pil.south west,anchor=center] {Costs};

    \draw[<->] (sigma) -- (pil);
    \draw[<->] (cost) -- (pil);

    \draw[->] (psl) -- (pvt);
    \draw[->] (pil) -- (pvt);
  \end{tikzpicture}
\caption{CACE typical workflow \cite{CACE}}
\label{fig:cace_workflow}
\end{figure}

\subsection{PIL}

PIL, the Protocol Implementation Language is a language for specifying
protocols. As a language on its own it has support for the following
features:
\begin{itemize}
\item global shared constants (parameters)
\item global constants (parameters)
\item global variables
\item local variables
\item conditionals
\item loops
\item functions
\item predicates
\end{itemize}

Proof entities are specified as blocks and there is always a Common
block with all declarations and definitions visible to all other
blocks. Each block can define multiple functions that have inputs and outputs
defined. A function consists of assignments or loops or conditional
flow. The execution order is specified via block function pairs.

\filbreak

\lstinputlisting[language=PIL]{example.pil}

\section{ZKPDL}

ZKPDL is a description language for describing Zero Knowledge Proofs
of Knowledge \cite{zkpdl}.

\begin{figure}[hbt!]
  \centering
  \begin{tikzpicture}[>=stealth,level distance=1.5cm,sibling distance=2cm,font=\tiny]
    \tikzstyle{edge from parent}=[draw,->]

    \Tree [.\node[language](zkpdl_prog){ZKPDL \\ Program};
      [.\node[compiler](compile){compile()};
        [.\node[compiler](prover){Intepreter \\ Prover};
          [.\node[language](message){ZKProof \\ Message};]
        ]
        [.\node[compiler](verifier){Interpreter \\ Verifier};
          [.\node[language](verified){Proof verified};]
        ]
      ]
    ]

    \node[language,font=\tiny](private)[left=of prover]{private values \\ to be proved};

    \node[language,font=\tiny](public)[left=of compile.north west]{public values};

    \draw[->] (public) -- (compile);
    \draw[->] (private) -- (prover);
    \draw[->] (message) -- (verifier);
  \end{tikzpicture}
  \caption{ZKPDL typical flow \cite{zkpdl}}
  \label{fig:zkpdl_flow}
\end{figure}

\section{Comparison of CACE and ZKPDL}

A comparison between CACE and ZKPDL is given in table
\ref{tab:compiler_comparison}.

\begin{table}[hbt!]
  \begin{tabularx}{0.33\textwidth}{|l|l|l|}
    & CACE compiler & ZKPDL compiler \\
    \hline \\
    Basic proof goals &
    \begin{minipage}{0.33\textwidth}
      knowledge of preimages under arbitrary group homomorphisms,
      including RSA or Paillier type homomorphisms
    \end{minipage}
    &
    \begin{minipage}{0.33\textwidth}
      exponentiation homomorphisms only      
    \end{minipage} \\
    \hline \\
    Basic protocols &
    \begin{minipage}{0.33\linewidth}
      SigmaPhi, Damgard/Fujisaki, SigmaExp
    \end{minipage}
    &
    \begin{minipage}{0.33\linewidth}
      SigmaPhi, Damgard/Fujisaki
    \end{minipage} \\
    \hline \\
    Composition techniques &
    \begin{minipage}{0.33\linewidth}
      Boolean AND, OR, and providing knowledge
      of $k$ out of $n$ secret values
    \end{minipage}
    &
    \begin{minipage}{0.33\linewidth}
      Boolean AND only
    \end{minipage} \\
    \hline \\
    Input language &
    \begin{minipage}{0.33\linewidth}
      inspired by the standard notation for ZK-PoK
    \end{minipage}
    &
    \begin{minipage}{0.33\linewidth}
      inspired by the English language
    \end{minipage} \\
    \hline \\
    Output language &
    \begin{minipage}{0.33\linewidth}
      C for implementation, \LaTeX for documentation
    \end{minipage}
    &
    \begin{minipage}{0.33\linewidth}
      specifically design language with interpreter and API to C++
    \end{minipage} \\
    \hline \\
    Type of output &
    \begin{minipage}{0.33\linewidth}
      complete source code, which can directly be compiled to machine
      code
    \end{minipage}
    &
    \begin{minipage}{0.33\linewidth}
      meta code or libraries, which partly have to be instantiated by
      the calling procedure
    \end{minipage} \\
    \hline \\
    Optimizations &
    \begin{minipage}{0.33\linewidth}
      reduction of redundant terms in the proof goal
    \end{minipage}
    &
    \begin{minipage}{0.33\linewidth}
      \begin{enumerate}
      \item reduction of redundant terms in the proof goal
      \item possibility of caching precomputations to reduce runtime
      \end{enumerate}
    \end{minipage} \\
    \hline \\
    Additional features &
    \begin{minipage}{0.33\linewidth}
      \begin{enumerate}
      \item existence of a formal verification toolbox, which proves
        the correctness of the compilation process
      \item modular design with interfaces to other cryptographic
        compilers developed within the CACE project
      \end{enumerate}
    \end{minipage}
    &
    \begin{minipage}{0.33\linewidth}
      possibility to specify the generation of protocol inputs
    \end{minipage}
  \end{tabularx}
  \caption{Detailed comparison of ZKPoK compilers \cite{yaczk}}
  \label{tab:compiler_comparison}
\end{table}

\section{ANTLR \cite{ANTLR,ANTLR2}}

ANTLR (ANother Tool For Language Recognition) is a parser generator
tool that generates LL(*) parsers. The tool accepts grammar
definitions as input files and produces output in the target language
which can be specified between C, Java or Python.

The input to ANTLR is a context-free grammar augmented with syntactic
and semantic predicates. Syntactic predicates allow arbitrary look-ahead
while semantic predicates allow the state constructed up to the point of
the predicate to direct the parse \cite{ANTLR}.

The lexer and parser grammars can be combined into a single file.  In
such cases, parser rules are written in lowercase, while the lexer
rules are written in uppercase. ANTLR workflow allows for tree
generators and tree walkers.

\section{LLVM}

LLVM (Low Level Virtual Machine) is a compiler framework designed to
support transparent, life-long program analysis and transformation
for arbitrary programs, by providing high-level information to
compiler transformations at compile-time, link-time, run-time, and in
idle time between runs \cite{LLVM:CGO04}.

Traditional compilers were tailored for only a few languages (with the
exception of GCC). However, all traditional compilers suffer from the
large inter-dependency of the basic blocks (Front-end, Optimizer,
Back-end). LLVM tries to solve this by providing an intermediate form
called the LLVM IR. A typical flow involving the basic blocks is
depicted in Figure \ref{fig:llvm_flow}.

\begin{figure}[hbt!]
  \centering
   \begin{tikzpicture}[>=stealth]
    \tikzstyle{lang}=[rectangle,draw=black,thin,font=\tiny,inner
    sep=0pt, align=center,minimum width=2.7cm,minimum height=2.2em]

    \tikzstyle{txt}=[font=\tiny]

    \node[lang](llvm_opt){LLVM \\ Optimizer};
    \node[lang](ppc_back)[right=1 cm of llvm_opt]{LLVM \\ PowerPC Backend};
    \node[txt](ppc)[right=of ppc_back]{PowerPC};
    \node[lang](x86_back)[above of=ppc_back]{LLVM \\ x86 Backend};
    \node[txt](x86)[right=of x86_back]{x86};
    \node[lang](arm_back)[below of=ppc_back]{LLVM \\ ARM Backend};
    \node[txt](arm)[right=of arm_back]{ARM};

    \node[lang](gcc_front)[left=1 cm of llvm_opt]{llvm-gcc \\ Frontend};
    \node[txt](fortran)[left=of gcc_front]{Fortran};
    \node[lang](clang_front)[above of=gcc_front]{Clang C/C++/ObjC \\ Frontend};
    \node[txt](c)[left=of clang_front]{C};
    \node[lang](ghc_front)[below of=gcc_front]{GHC \\ Frontend};
    \node[txt](haskell)[left=of ghc_front]{Haskell};

    \draw[->] (clang_front.east) -- (llvm_opt.west);
    \draw[->] (gcc_front.east) -- (llvm_opt.west);
    \draw[->] (ghc_front.east) -- (llvm_opt.west);

    \draw[->] (llvm_opt.east) -- (x86_back.west);
    \draw[->] (llvm_opt.east) -- (ppc_back.west);
    \draw[->] (llvm_opt.east) -- (arm_back.west);

    \draw[->] (x86_back) -- (x86);
    \draw[->] (ppc_back) -- (ppc);
    \draw[->] (arm_back) -- (arm);

    \draw[->] (c) -- (clang_front);
    \draw[->] (fortran) -- (gcc_front);
    \draw[->] (haskell) -- (ghc_front);
  \end{tikzpicture}
  \caption{LLVM typical workflow}
  \label{fig:llvm_flow}
\end{figure}

Due to its modularity, LLVM has recently seen increased usage in a number
of independent fields:
\begin{itemize}
\item implementing a C/C++ compiler (Clang)
\item implementing C-to-HDL translation
\item implementing a Haskell compiler (GHC)
\item implementing Secure Virtual Architectures (SVA)
\item implementing dynamic translation
\item implementing OpenGL drivers (Mac OS X)
\item implementing OpenCL drivers (AMD)
\end{itemize}

\subsection{LLVM IR}

The LLVM IR is the intermediate representation language of the LLVM
project. On its own it is a first-class language with well defined
semantics \cite{llvm_general}. It resembles the assembly language of a
infinitely many registers Load-Store based RISC processor.

\begin{lstlisting}[language=C]
unsigned add1(unsigned a, unsigned b) {
  return a+b;
}

// Perhaps not the most efficient way to add two numbers.
unsigned add2(unsigned a, unsigned b) {
  if (a == 0) return b;
  return add2(a-1, b+1);
}
\end{lstlisting}

\begin{lstlisting}
define i32 @add1(i32 %a, i32 %b) {
entry:
  %tmp1 = add i32 %a, %b
  ret i32 %tmp1
}

define i32 @add2(i32 %a, i32 %b) {
entry:
  %tmp1 = icmp eq i32 %a, 0
  br i1 %tmp1, label %done, label %recurse

recurse:
  %tmp2 = sub i32 %a, 1
  %tmp3 = add i32 %b, 1
  %tmp4 = call i32 @add2(i32 %tmp2, i32 %tmp3)
  ret i32 %tmp4

done:
  ret i32 %b
}
\end{lstlisting}

\filbreak

The central concept in constructing LLVM IR is the Module. The Module
represents the top level container of the LLVM IR. There are multiple
ways on how to generate the IR. The one described here is by building a
graph and creating objects along that graph.

\section{GEZEL}

GEZEL is a cycle-accurate hardware description language (HDL) using the
Finite-State-Machine + Datapath (FSMD) model.

%%% Local Variables: 
%%% TeX-PDF-mode: t
%%% TeX-master: "thesis"
%%% End: 
