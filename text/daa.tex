\chapter{Use Case: Direct Anonymous Attestation}

This chapter will attempt to show the advantages of the custom
framework by implementing a simplified Direct Anonymous Attestation
(DAA) Join protocol taken from \cite{simplified_daa}. The reason Join
was chosen is that it is not a $\Sigma$ protocol and cannot be
implemented using the CACE Project Zero Knowledge Compiler. Join
protocol also serves as an example of a more complex and specific
protocol as it is a multiparty protocol. Again, the benefit
of the custom framework is that it allows the input of basic PIL as
well. The Sign protocol is not of interest here.


\section{Join protocol}

The Join protocol is the first phase in the DAA protocol. Its goal is
to establish the certificate. The protocol is shown in figure
\ref{fig:daa_join}. The protocol is not a Sigma protocol and cannot be
specified using the PSL language. Also, the protocol uses 3 parties
which the CACE Project Zero Compiler does not support. This, and
similar multiparty protocols were the reason for extensions to the PIL
language (see Sub-section \ref{subsec:pil_extensions}) in the first
place. It is observable how the specification in the PIL language
closely follows the protocol flow diagram.

\begin{figure}[H]
  \centering
  \begin{tikzpicture}[>=stealth]
    \node[matrix,column sep=1cm] {
      \node{Smartcard};                       &                           & \node{Host}; &                             & \node{Credential Issuer}; \\
      \node{$f \in \{0,1\}^{l_f+l_\phi+l_H+2}$};   &                           &              &                             &                   \\
      \node{$v' \in \{0,1\}^{l_f+l_\phi+l_H+2}$};  &                           &              &                             &                   \\
      \node{$U = \pm R^f S^{v'} \pmod{n}$};     & \node(scard_round1_s){};  &              & \node(cred_round1_r){};     & \\
                                               &                           &              &                             & \node{$\hat{v} \in \{0,1\}^{l_v-1}$}; \\
                                               &                           &              &                             & \node{$v'' = \hat{v} + 2^{l_v-1}$}; \\
                                               &                           &              &                             & \node{$e \in \left[ 2^{l_e-1}, 2^{l_e-1} + 2^{l_{e'}-1} \right]$}; \\
                                               &                           &              &                             & \node{$A = \left( \frac{Z}{U S^{v''}} \right)^{\frac{1}{e}} \pmod{n}$}; \\
      \node{$v''$};                            & \node(scard_round2_r){};  &              & \node(cred_round1_s){};     & \node{$(A, e, v'')$};\\
      \node{$v = v' + v''$};                   &                           &              &                             & \\
    };
    \draw[->] (scard_round1_s) -- (cred_round1_r);
    \draw[<-] (scard_round2_r) -- (cred_round1_s);
  \end{tikzpicture}
  \caption{DAA Join protocol}
  \label{fig:daa_join}
\end{figure}

It is also visible how the usage of constant expressions aids the
writer in specifying the protocol. It suffices to only change the
parameter and the constant expressions will recompute the sizes
according to the expression.

\lstinputlisting[language=PIL]{Join.pil}

%%% Local Variables:
%%% TeX-PDF-mode: t
%%% TeX-master: "thesis"
%%% End:
