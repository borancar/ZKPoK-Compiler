\chapter{Use Case: Schnorr's Identification Protocol on a Custom System}

This chapter aims to present a use case of our framework and the
possibilities for extension. We take Schnorr's Identification Protocol
and implement it on a system not unlike today's smart cards. This
requires us to write a custom back-end for our framework and here we
detail the steps.

\section{Motivation}

So far, two extremes of the HW-SW co-design spectrum have been
explored using our framework and we think there is little need to
justify an exploration of the spectrum somewhere in the ``middle''.
Instead of generating our own system we take an existing system and
demonstrate the flexibility and adaptability of our framework.

The target system is not unlike today's smart cards in the manner that
it uses a small and cheap micro-controller coupled with a custom
co-processor to offload the more demanding calculations. We have
already discussed the advantages of Zero Knowledge Proofs of Knowledge
over traditional policy and authentication verification (see Section
\ref{sec:thesis_motivation}) and we deem the Schnorr's Protocol simple,
yet concise enough to demonstrate the applicability of such protocols
and our framework.

\section{Target System}

The system targeted by the framework is an existing system developed
within a course here at KU Leuven. The system includes a general
purpose micro-controller, 8051 in this case, coupled with a custom
designed cryptographic co-processor (see Appendix
\ref{cha:codesign_project}).

The co-processor implements the Montgomery multiplication as the
project deemed this the most critical operation that needed a hardware
implementation. The operations of modular multiplication and modular
exponentiation are handled in software using the Montgomery
multiplication as a building block.

Command queuing (see Appendix \ref{cha:codesign_project}) was used as
a technique to compensate for the slower execution speed of the main
processor. It also allows the main processor to perform other tasks
while the co-processor is executing the uploaded instructions.

\section{Implementation Flow Example}
\label{sec:usecase_flow}

First we start with the PSL implementation of Schnorr's Identification
Protocol (already given in Sub-section \ref{subsec:psl}). This PSL
file is given to the CACE ZKC (see Section \ref{sec:cace}) to produce
the resulting PIL and C files. The C files are compiled to produce an
actual implementation of a prover and a verifier which are then
configured and tested on a general purpose computer. This part is
equivalent to the CACE ZKC flow.

Next, our extension takes the resulting Schnorr PIL file and compiles
it into an intermediate form capturing the operations within it. As
the target does not support modular multiplication and modular
exponentiation, intrinsic lowering is performed, adding the necessary
transfer to and from the Montgomery domain. Since this step would
generate a lot of redundant intermediate results, a special
optimization step is run to eliminate these intermediate results.

\section{Results and Possible Improvements}
\label{sec:usecase_results}

\subsection{Results}

The entire protocol run requires around $\unit[2]{mil.}$ cycles, which
is roughly $\unit[0.5]{s}$ running at the proposed frequency of
$\unit[4]{MHz}$.

\subsection{Further Improvements}

The approach taken was to reuse the software library provided. We note
that the ideal goal step would be to emit the required instructions
directly and such a step is not difficult as we are dealing with LLVM,
a compiler infrastructure framework. Emitting the instructions
directly instead of using library calls can yield a further
optimization as the results are transferred less from the co-processor
to the main processor. Here we omitted this step as the gains were too
little and we deemed the final compilation not to be a key point of
this thesis.

%%% Local Variables: 
%%% TeX-PDF-mode: t
%%% TeX-master: "thesis"
%%% End: 
