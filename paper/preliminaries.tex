

\subsection{ZKPK $\Sigma$-protocols}

A $\Sigma$-protocol~\cite{DamgardSigma2004} is a three round honest-verifier ZKPK where a prover P proves to a verifier V that a secret $x$ satisfies a statement with respect to a public $y$.  The rounds of the protocol are as follows. First, on input $(x,y)$, P computes a commitment $t$ and auxiliary information $r$, and sends $t$ to V. Second, V sends a random challenge $c$ to P. Third, on input $(x,y,c,r)$, P computes a response $s$ and sends $s$ to V. Finally, on input $(x,y,c,s)$, V checks whether the proof is correct. 

A simple example of a $\Sigma$ protocol is Schnorr's Identification
Protocol~\cite{schnorr_protocol} depicted in Figure~\ref{fig:schnorr_steps}. The
secret is the discrete logarithm $x$ of $y$ with respect to $g$ in some
finite group $G = \left< g \right>$ with prime order $q$, subgroup of
$Z_p^*$. The prover must prove knowledge of this secret to a verifier under
the homomorphism $f(a) : Z_q^+ \rightarrow G \subset Z_p^* := g^a$.
\begin{figure}[hbt!]
  \centering
  \begin{tikzpicture}[>=stealth]
    \node[matrix,column sep=1cm] {
      \node{\boxed{Prover}};                  &                           & &                             & \node{\boxed{Verifier}}; \\
      \node{$r \in Z_q^+$};                   &                           & &                             &                   \\
      \node{$t = g^{r}$};          & \node(prover_round1_s){}; & & \node(verifier_round1_r){}; & \\
                                      & \node(prover_round2_r){}; & & \node(verifier_round1_s){}; & \node{$c \in \{0,1\}^*$}; \\
      \node{$s = r + x \cdot c$}; & \node(prover_round2_s){}; & & \node(verifier_round2_r){}; & \node{$s \stackrel{?}{\in} Z_q^+$}; \\
                                      &                           & &                             & \node{$t \cdot y^c \stackrel{?}{=} g^{s}$}; \\
    };
    \draw[->] (prover_round1_s) -- (verifier_round1_r) node[midway, anchor=south]{$t$};
    \draw[<-] (prover_round2_r) -- (verifier_round1_s) node[midway, anchor=south]{$c$};
    \draw[->] (prover_round2_s) -- (verifier_round2_r) node[midway, anchor=south]{$s$};
  \end{tikzpicture}
  \caption{Schnorr's Identification Protocol}
  \label{fig:schnorr_steps}
\end{figure}


A simple and popular notation for specifying $\Sigma$-protocols is the
Camenisch-Stadler notation~\cite{camenisch_stadler}. The Schnorr
Identification Protocol can be represented by the following:
\[
  \textrm{ZKPK}\left[ (x): y = g^x \right]
\]
meaning prove knowledge of $x$, using a zero knowledge proof of
knowledge, such that $y=g^x$.

Protocols similar to~\cite{schnorr_protocol} that employ groups of hidden-order have been proposed in~\cite{DBLP:conf/crypto/FujisakiO97,DBLP:journals/iacr/BangerterCKS008}. In addition, $\Sigma$-protocols can be composed to obtain more complex protocols. The AND-composition allows to prove, e.g., statements of the form $y_1 = g_1^{x_1} \land \ldots \land y_n = g_n^{x_n}$. OR-composition and threshold composition are also available. Additionally, it is possible to prove equality and linear relations between the secrets $(x_1,\ldots,x_n)$.

A $\Sigma$-protocol should fulfill the special soundness and special honest-verifier zero-knowledge properties~\cite{DamgardSigma2004}. The Fiat-Shamir heuristic~\cite{DBLP:conf/crypto/FiatS86} shows how to turn a $\Sigma$-protocol into a non-interactive ZKPK in the random oracle model.

\subsection{Data Flow Graph and Control Flow Graph}

\begin{defn}[Data Edge]
  A data edge is an ordered relation between two operations such that
  the output/result of one operation is the input to the other
  operation.
\end{defn}

\begin{defn}[Control Edge]
  A control edge is an ordered relation between two operations such
  that the second operation has to be executed after the first finishes
  executing.
\end{defn}

\begin{defn}[Data Flow Graph]
  A data flow graph is a graph of the nodes representing the
  operations connected by data edges.
\end{defn}

\begin{defn}
  A control flow graph is a graph of the nodes representing the
  operations connected by control edges.
\end{defn}

An algorithm is completely and uniquely defined by its Data Flow Graph
(DFG), whereas the implementation of an algorithm is what gives it its
Control Flow Graph (CFG). The usual step in converting an algorithm in
software to an algorithm in hardware is to extract the DFG
\cite{Schaumont}.

%\subsection{$\Sigma$-protocols}
%
%A $\Sigma$-protocol~\cite{DamgardSigma2004} is a three round honest verifier ZKPK. The name comes
%from the protocol's flow resemblance to the Greek letter $\Sigma$ as
%shown in Figure \ref{fig:sigma_flow}. The rounds of the protocol are:
%\begin{enumerate}
%\item Commitment ($t$) - the prover commits to a value and sends that
%  value to the verifier
%\item Challenge ($c$) - the verifier computes a random challenge and asks
%  the prover to output a value for that challenge
%\item Response ($s$) - the prover responds with a new computed value
%\end{enumerate}
%
%\begin{figure}[hb!]
%  \centering
%  \begin{tikzpicture}[>=stealth]
%    \node[matrix,column sep=1cm] {
%      \node{\boxed{Prover}};                  &                           & &                             & \node{\boxed{Verifier}}; \\
%      \node{Commitment};          & \node(commitment_s){}; & & \node(commitment_r){}; & \\
%                                      & \node(challenge_r){}; & & \node(challenge_s){}; & \node{Challenge}; \\
%      \node{Response}; & \node(response_s){}; & & \node(response_r){}; & \\
%    };
%    \draw[->] (commitment_s) -- (commitment_r) node[midway, anchor=south]{$t$};
%    \draw[<-] (challenge_r) -- (challenge_s) node[midway, anchor=south]{$c$};
%    \draw[->] (response_s) -- (response_r) node[midway, anchor=south]{$s$};
%  \end{tikzpicture}
%  \caption{Sigma protocol flow}
%  \label{fig:sigma_flow}
%\end{figure}
%
%\subsubsection{Schnorr's Identification Protocol}
%\label{subsubsec:schnorr_protocol}
%
%A simple example of a $\Sigma$ protocol is Schnorr's Identification
%Protocol \cite{schnorr_protocol,cryptography_introduction}. The
%secret is a discrete logarithm $x$ of $y$ with respect to $g$ in some
%finite group $G = \left< g \right>$ with prime order $q$, subgroup of
%$Z_p^*$. The prover must prove knowledge of this secret to a verifier under
%the homomorphism $f(a) : Z_q^+ \rightarrow G \subset Z_p^* := g^a$.
%
%\begin{figure}[hbt!]
%  \centering
%  \begin{tikzpicture}[>=stealth]
%    \node[matrix,column sep=2cm] {
%      \node{Prover};                  &                           & &                             & \node{Verifier}; \\
%      \node{$r_1 \in Z_q^+$};                   &                           & &                             &                   \\
%      \node{$t_1 = g^{r_1}$};          & \node(prover_round1_s){}; & & \node(verifier_round1_r){}; & \\
%                                      & \node(prover_round2_r){}; & & \node(verifier_round1_s){}; & \node{$c \in \{0,1\}^*$}; \\
%      \node{$s_1 = r_1 + x \cdot c$}; & \node(prover_round2_s){}; & & \node(verifier_round2_r){}; & \\
%                                      &                           & &                             & \node{$s_1 \stackrel{?}{\in} Z_q^+$}; \\
%                                      &                           & &                             & \node{$t_1 y^c \stackrel{?}{=} g^{s_1}$}; \\
%    };
%    \draw[->] (prover_round1_s) -- (verifier_round1_r) node[midway, anchor=south]{$t_1$};
%    \draw[<-] (prover_round2_r) -- (verifier_round1_s) node[midway, anchor=south]{$c$};
%    \draw[->] (prover_round2_s) -- (verifier_round2_r) node[midway, anchor=south]{$s_1$};
%  \end{tikzpicture}
%  \caption{Schnorr's Identification Protocol}
%  \label{fig:schnorr_steps}
%\end{figure}
%
%\subsubsection{Camenisch-Stadler Notation}
%\label{subsubsec:camenisch_stadler}
%
%A simple and popular notation for specifying Sigma protocols is the
%Camenisch-Stadler notation \cite{CACE,camenisch_stadler}. The Schnorr
%Identification Protocol can be represented by the following:
%\[
%  \textrm{ZPK}\left[ (x): y = g^x \right]
%\]
%meaning prove knowledge of $x$, using a zero knowledge proof of
%knowledge, such that $y=g^x$.
%
%\subsection{Syntax Trees and Parsers}
%
%\begin{defn}[Grammar]
%  A grammar is a 4-tuple $G = \left< \Sigma, V, S, P \right>$:
%  \begin{enumerate}
%  \item $\Sigma$ is a finite non-empty set called the \emph{terminal
%    alphabet}. The elements of $\Sigma$ are called \emph{terminals}.
%  \item $V$ is a finite non-empty set disjoint from $\Sigma$. The
%    elements of $V$ are called \emph{non-terminals} or \emph{variables}.
%  \item $S \in V$ is a distinguished symbol called the \emph{start symbol}
%  \item $P$ is a finite set of \emph{productions (rules)} of the form
%    $$ \alpha \rightarrow \beta $$
%    $$ \alpha \in (\Sigma \cup V)^* V (\Sigma \cup V)^* $$
%    $$ \beta \in (\Sigma \cup V)^* $$
%  \end{enumerate}
%\end{defn}
%
%\begin{defn}[Context-free grammar]
%  The grammar $G$ is a context free grammar iff $|\alpha|$ = 1 ($\alpha = V$).
%\end{defn}
%
%\subsection{Concrete Syntax Tree}
%
%A Concrete Syntax Tree, also called a Parse Tree, is an ordered tree
%representation of the input according to a given formal grammar.
%
%For example, given the following input:
%\begin{lstlisting}
%a := b * c + d;
%\end{lstlisting}
%and the following grammar:
%\begin{align*}
%  \textrm{statement}  & \rightarrow \textrm{ID} := \textrm{expression} ; \\
%  \textrm{expression} & \rightarrow \textrm{term} + \textrm{term} \\
%  \textrm{term}       & \rightarrow \textrm{ID} * \textrm{ID} \\
%  \textrm{term}       & \rightarrow \textrm{ID}
%\end{align*}
%the tree depicted in Figure \ref{fig:simple_cst} is a Concrete Syntax Tree.
%
%\begin{figure}[hbt!]
%  \centering
%  \begin{tikzpicture}
%    \Tree [.\node[language](statement){statement};
%      [.\node[language](a){a};]
%      [.\node[language](assign){:=};]
%      [.\node[language](expression){expression};
%        [.\node[language](term){term};
%          [.\node[language](b){b};]
%          [.\node[language](mul){*}; ]
%          [.\node[language](c){c};]
%        ]
%        [.\node[language](plus){+};]
%        [.\node[language](term2){term};
%          [.\node[language](d){d};]
%        ]
%      ]
%      [.\node[language](semi){;};]
%    ]
%  \end{tikzpicture}
%  \caption{Concrete Syntax tree for a simple input and a simple grammar}
%  \label{fig:simple_cst}
%\end{figure}
%
%\subsection{Abstract Syntax Tree}
%
%An Abstract Syntax Tree (AST) is a tree representation of the abstract
%syntax of the input. Given the same example
%\begin{lstlisting}
%a := b * c + d;
%\end{lstlisting}
%one possible variant of the tree is illustrated in Figure
%\ref{fig:simple_ast}. Comparing it with the Parse Tree from Figure
%\ref{fig:simple_cst} one can see that the choice of abstraction is
%arbitrary.
%
%\begin{figure}[hbt!]
%  \centering
%  \begin{tikzpicture}
%    \Tree [.\node[language](equ){=};
%      [.\node[language](a){a};]
%      [.\node[language](add){+};
%        [.\node[language](mul){*};
%          [.\node[language](b){b};]
%          [.\node[language](c){c};]
%        ]
%        [.\node[language](d){d};]
%      ]
%    ]
%  \end{tikzpicture}
%  \caption{An example AST for a simple expression}
%  \label{fig:simple_ast}
%\end{figure}
%
%\subsection{Parser}
%
%Given a string $L$, satisfying grammar $G$, a parser tries to find the
%derivation of $L$ from $S$. This derivation is the Concrete Syntax
%Tree (Parse Tree). This derivation may further be abstracted into an
%Abstract Syntax Tree that is easier for manipulation later. Figure
%\ref{fig:parser_flow} shows the typical parser flow.
%
%\begin{figure}[hbt!]
%  \centering
%  \begin{tikzpicture}[>=stealth,level distance=1.5cm,font=\tiny]
%    \tikzstyle{edge from parent}=[draw,->]
%
%    \Tree [.\node[language](input){Input};
%      [.\node[compiler](parser){Parser};
%        [.\node[language](ptree){Parse tree};
%          [.\node[compiler](tgen){Tree generator};
%            [.\node[language](ast){Abstract Syntax Tree};
%            ]
%          ]
%        ]
%      ]
%    ]
%  \end{tikzpicture}
%  \caption{Parser flow}
%  \label{fig:parser_flow}
%\end{figure}
%
%There are two approaches to parsing:
%\begin{description}
%\item[Top-down parsing] the derivation starts from the top (the root)
%  of the parse tree and proceeds downwards.
%\item[Bottom-up parsing] the derivation starts from the bottom (the
%  leaves) of the parse tree and proceeds upwards.
%\end{description}
%
%The most popular top-down parser is an LL (left to right left-most
%derivation) parser. The parser reads the input left to right and at
%each step produces the left-most derivation of the input. Sub-types of
%this parser include the symbol look-ahead which is the amount of next,
%unseen symbols the parser can ``look at''. For example, an LL(1)
%parser can only look 1 symbol ahead while an LL(*) parser is a parser
%with arbitrary look-ahead.
%
%
%\section{Data Flow Graph and Control Flow Graph}
%
%\begin{defn}[Data Edge]
%  A data edge is an ordered relation between two operations such that
%  the output/result of one operation is the input to the other
%  operation.
%\end{defn}
%
%\begin{defn}[Control Edge]
%  A control edge is an ordered relation between two operations such
%  that the second operation has to be executed after the first finishes
%  executing.
%\end{defn}
%
%\begin{defn}[Data Flow Graph]
%  A data flow graph is a graph of the nodes representing the
%  operations connected by data edges.
%\end{defn}
%
%\begin{defn}
%  A control flow graph is a graph of the nodes representing the
%  operations connected by control edges.
%\end{defn}
%
%An algorithm is completely and uniquely defined by its Data Flow Graph
%(DFG), whereas the implementation of an algorithm is what gives it its
%Control Flow Graph (CFG). The usual step in converting an algorithm in
%software to an algorithm in hardware is to extract the DFG
%\cite{Schaumont}.

%%% Local Variables:
%%% TeX-PDF-mode: t
%%% TeX-master: "paper"
%%% End:
