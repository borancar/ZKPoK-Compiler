We have proposed a ZKPK compiler that takes an implementation in PIL
as output by the CACE ZKPK Compiler and produces implementations in
GEZEL and GMP, thus covering both ends of the hardware-software
co-design spectrum. We consider that this is a good starting point for
a fine-grained automated approach to hardware-software co-design.

We support all the protocols the CACE ZKPK compiler
supports. Additionally, we have extended the PIL language to allow for
multiparty protocols and we have also restricted the language
specification by allowing only the currently running round to modify
global variables. The latter has shown that the PIL language is
expressive enough, yet still suitable for easy DFG extraction.

\subsection{Future work}

\paragraph{Extending LLVM IR with Group Types.}
The current implementation has to back all the group types with a
simplified type that loses information. We consider that implementing
group types in LLVM IR should enhance the usability and security of
the IR by allowing the back-end access to the full information of
a group type.

\paragraph{HW-SW Co-design.}
Another step that should be taken is the automated HW-SW co-design via
a general purpose processor connected with a custom designed
co-processor. The tool should allow the user to separate the features
he wants on the processor and the features he wants on the
co-processor side.

\paragraph{Automatic Generation of Group Type Variables.}
CACE ZKPK compiler does not allow for automatic generation of primes
or group generators, but takes them from a configuration file. We have
extended PIL with constant expressions so that they can be used for
this type of automatic generation. Specialized generator functions
need to be implemented.

%\paragraph{A Common Lower-level Language for the Entire CACE Project.}
%Out of all the ZPKP compiler frameworks, the CACE Project was deemed
%the most versatile to base our work on. We have noted that each of the
%CACE Project workpackages designed their own lower end language and
%have limited themselves with the architectures they support. It is our
%belief that an extended LLVM IR can be of good use here, providing a
%unified lower level language and allowing the generated code of the
%different tools to be intermixed.

%%% Local Variables:
%%% TeX-PDF-mode: t
%%% TeX-master: "paper"
%%% End:
