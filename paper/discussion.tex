
\subsection{Optimizations}

LLVM provides a constant folder which allows constant expressions to
be evaluated at compile time. This eliminates all the redundant calls
that would otherwise be needed at run-time, which can reduce the number
of complex, costly arbitrary-precision operations.

LLVM IR is of the static single assignment form, which allows
aggressive optimizations such as dead-code elimination. Another
advantage of LLVM is the possibility of separating the optimizations
to higher level (common back-end) and lower level (specific back-end).

Since accessing the global variables is slower, writing to the global
variable can be left for the end of the current round or deferred to a
later time, when needed. This allows optimizations in a
hardware-software co-design world where it can be expensive to move
data around.  The optimization is achieved by traversing all the load
instructions that follow a store instruction.  If any of those load
instructions loads from the same location which a store location wrote
to, all the nodes that are dominated by the loaded value are replaced
with the value that was stored. In our model of computation this is
perfectly sane since only the current round can modify a global
variable.

Another optimization example we can think of is exponentiation via
Montgomery multiplications. If we were to translate all the operations
of the algorithm step-by-step, we would have a transfer to the
Montgomery domain, multiplication in the Montgomery domain and return
from the Montgomery domain. The high-level optimizer can eliminate the
unnecessary transfers and returns since they are inverse operations
performed consecutively.

\subsection{Security Analysis}

CACE ZKPK Compiler already offers a formal proof of correctness
covering the transformation from PSL to PIL. We need to analyze the
security of the transformation from PIL to LLVM IR and LLVM IR to the
back-ends whenever possible.

LLVM IR allows us to verify the correctness of the types since LLVM IR
uses code of the static single assignment form, where every variable
is assigned a value exactly once. This allows security assertions to
be taken to a lower level than the one allowed by CACE. 

To assure the correctness of types, first it is necessary to lay out
the DFG of the protocol round. For each node of the DFG preconditions
and postconditions cab be established, such as the ones in Table
\ref{tab:basic_nodes}. Hoare logic~\cite{hoare_logic} can then be
used to prove the correctness of the transformation from PIL to the
LLVM IR.

As for the transformation of LLVM IR into target implementations, it
might not always be possible to prove correctness, especially if the
target implementation is hardware. Since it is impossible to make this
assurance it was deemed unnecessary to go this low and the blocks are
assumed to be correct (if the preconditions are satisfied, the
postconditions will be assured by the block). For the software case,
such properties are assured by the library used.

\begin{table}[h!]
  \centering
  \begin{tabular}{l | c | c}
    Basic block        & Preconditions             & Postconditions \\
    \hline
    Modular adder      & $x \in Z_q^+, y \in Z_q^+$ & $z \in Z_q^+$ \\ 
    Modular multiplier & $x \in Z_p^*, y \in Z_p^*$ & $z \in Z_p^*$ \\
    Zero extender      & $x \in Z_p$               & $z \in Z_p$
  \end{tabular}
  \caption{Basic Nodes}
  \label{tab:basic_nodes}
\end{table}

%%% Local Variables: 
%%% TeX-PDF-mode: t
%%% TeX-master: "main"
%%% End: 
